%%==================================================
%% conclusion.tex for SJTU Master Thesis
%% based on CASthesis
%% modified by wei.jianwen@gmail.com
%% version: 0.3a
%% Encoding: UTF-8
%% last update: Dec 5th, 2010
%%==================================================

\chapter*{总结与展望\markboth{总结与展望}{}}
\addcontentsline{toc}{chapter}{总结与展望}


\section{本文主要工作}

本文的主要工作,是研究软管组件的纤维编织加强层的力学本构模型。

本研究的工作主要集中在3各方面:

1.理论分析和探索。基于Hachami 的金属编织层本构理论,以及拉伸试验的结果,分析编织层结构中影响编织层,使其产生非线性力学行为的因素,及其影响程度。
选择因为该理论能够有效的结合加强层理论体系中的基于“钢绞线理论”的方法以及基于复合材料的方法。



2.实验操作和分析。对不同尺寸金属编织软管进行了拉伸实验。探索了一套完整的实验测试方法。

3.数值仿真。包括了对拉伸实验的仿真,以及对内压爆破实验的仿真。仿真采用通用有限元软件\aba 开放的二次开发子程序端口\uma 。






\section{本文主要贡献}

本文的主要贡献集中在以下几个创新点:

1.新的研究对象。尽管软管组件的纤维编织加强层已出现超过半个世纪,其指导设计的理论仍然是简单“薄壁圆柱”的受力平衡,经验公式以及实验;区别普通的纤维加强符合材料,编织加强层没有树脂相,是一种柔性的、表现出结构非线性力学行为的结构。

2.编织加强层非线性段的力学行为分析。由于这种结构的复杂性,以往无论基于解析还是数值的分析方法都么有进入到编织层发生大变形,产生大非线性的力学行为阶段。

3.修正刚度的方法。许多复合材料力学常见的修正方法第一次被引进编织加强层的理论体系中。而以往的编织层理论都是过度简化,不能考虑纤维间复杂相互关系的。

4.仿真内压爆破的方法。内压爆破试验是设计编制加强层 的核心试验,仿真内压荷载需要主要是为了确定内压荷载与加强层纤维应力的关系,也需要一套可行的强度理论。基于准静态的仿真,结合材料力学中突加荷载理论,提出了一套可以准确仿真内压爆破试验的方法。



\section{下一步工作展望}

1.本研究中尚未针对非金属编织层进行有关的实验。理论上本文的理论体系可以应用于非金属纤维编织层。但模型中的一些修正的处理,只在金属纤维的计算中进行过验证,不能保证非金属纤维也需要这些修正。

2.本研究的模型尚未引入金属纤维的塑性。其一,是因为塑性只能针对金属编织层,非金属纤维一般不会出现塑性的情况;其二,是因为通过实验结果判定金属纤维产生塑性的区域非常小,而高强度的冷拉钢丝断裂伸长率也非常小,塑性对拉伸的结果影响不会很大。将来对该模型进行扩展是,若涉及到容易产生塑性的纤维,这方面的研究也是很有意义的。

3.本研究的模型实际上只研究了单层编织层的力学本构,没有考虑多层加强层之间的相互影响,也没有研究内管和编织层之间的相互作用,这方面的研究是将来设计生产高压软管组件编织加强层的核心问题之一,非常值得进一步研究。

4.本研究目前只能应用于准态的分析,如何在编织层的本构中引入动态荷载的影响也是一个的难点。进一步引入分析软管组件编织层疲劳和寿命的方法也值得进一步探索。

