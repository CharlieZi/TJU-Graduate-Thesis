%%==================================================
%% thanks.tex for SJTU Master Thesis
%% based on CASthesis
%% modified by wei.jianwen@gmail.com
%% version: 0.3a
%% Encoding: UTF-8
%% last update: Dec 5th, 2010
%%==================================================

\begin{thanks}
\zihao{4}\hangju{1}
感谢上海市塑料研究所企事业委托项目
《某型PTFE金属编织增强软管组件结构设计原理解析及数值分析》对本文研究工作的支持。本文是在郑百林教授、贺鹏飞教授的悉心指导下完成的。正值毕业之际,我要感谢在攻读硕士学位期间给予我无私帮助和关怀的人。

首先我要感谢导师郑百林教授,在跟随郑老师的三年时光中,郑老师严谨而开放的培养风格,大大提升了我的学术研究能力、独立创新能力与社会阅历。郑老师毫无保留地与我分享他丰富的生活和科研经验,指引我将攻克学海难题的态度融入进生活之中,将乐观、积极的生活态度融入学习研究中,带领我度过了忙碌充实、斗志昂扬的三年,使我每天都能在“绞尽脑汁”后重获新生。

感谢导师贺鹏飞教授,贺老师学识渊博、视野雄阔,三年中不断地为我带来学科与产业的最前沿的研究与见解,这对于我来说是一笔非常宝贵的财富,培养了我对学术浓厚的兴趣。贺老师还为我学术和人生的规划提出了宝贵的意见,使我受益至今。没有您的指点,我很有可能会走上一条完全不同的生活道路。

感谢嵇老师对我学术工作的关心,您在讨论会中的指点对我有着极其重要的影响;感谢周仕刚老师,您对我实验的指导,使得本文能够顺利成文;感谢杨青博士马朋升博士对我科研学习的帮助;感谢张锴博士、马朋升博士对本文研究从开展到进行全心全力的支持与帮助;感谢李泳博士、从曙光、鲍照、王琪、何旅洋、杨彪、申洋、惠志鑫等同门师兄弟对我学习研究的支持,感谢倪静对我工作生活中的帮助与照顾,感谢已毕业的杨晓东博士、刘志刚、韩聪聪、朱建新、刘书、刘婧姝等师兄师姐对我的照顾与支持!

最后,感谢家人对我攻读硕士的支持,感谢父母默默支持我走过了二十多年的学习生涯!

\end{thanks}
